\documentclass{article}

\usepackage{array}
\usepackage{amsmath, amssymb, amsfonts} 
\usepackage[english]{babel}
\usepackage{booktabs}
\usepackage{enumitem}
\usepackage[margin=0.75in]{geometry}
\usepackage{graphicx}
\usepackage[colorlinks = true, linkcolor = blue, urlcolor = blue,
            citecolor = blue, anchorcolor = blue]{hyperref}
\usepackage[utf8]{inputenc}
\usepackage{makecell}
\usepackage{tabularray}
\usepackage{wrapfig}

\title{Paley Constructions of Hadamard Matrixes}
\author{Keith Wannamaker}

\begin{document}
\maketitle

\begin{abstract}
This document presents examples of constructing Hadamard matrixes
using two methods attributed to Raymond Paley.  In particular
the illustrations include quadratic residue calculations from a
Galois field for both the $p^1$ and $p^k$ cases, polynomial
division with remainder, and construction of the prerequisite
Jacobsthal matrix for both cases.

The construction methods are implemented in Java at
\url{https://github.com/wannamak/hadamard/}.
\end{abstract}

\section{Choosing a construction}

\begin{table}[htbp]
    \centering
    \caption{Paley Constructions}
    \begin{tblr}{colspec = { c Q[c,1cm] Q[c,1cm] },
                 % vlines,
                 hline{1,2,Z} = {1pt},  % <---
                 colsep=4pt,
                 }
    \thead{Hadamard\\order} & $p^k$ & \thead{Paley\\type}\\
      4 &  3 &  I \\
      8 &  7 &  I \\
     12 & 11 &  I \\
     16 &    &    \\
     20 & 19 &  I \\
     24 & 23 &  I \\
     28 & 13 & II \\
     32 & 31 &  I \\
     36 & 17 & II \\
     40 &    &
     44 & 43 &  I \\
    \end{tblr}
\end{table}

% \section*{Mission Statement}

\end{document}
