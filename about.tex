\documentclass{article}

\usepackage{array}
\usepackage{amsmath, amssymb, amsfonts} 
\usepackage[english]{babel}
\usepackage{booktabs}
\usepackage{enumitem}
\usepackage[margin=0.75in]{geometry}
\usepackage{graphicx}
\usepackage[colorlinks = true, linkcolor = blue, urlcolor = blue,
            citecolor = blue, anchorcolor = blue]{hyperref}
\usepackage[utf8]{inputenc}
\usepackage{makecell}
\usepackage{tabularray}
\usepackage{wrapfig}

\title{Paley Constructions of Hadamard Matrixes}
\author{Keith Wannamaker}

\begin{document}
\maketitle

\begin{abstract}
This document presents examples of constructing Hadamard matrixes
using two methods attributed to Raymond Paley.  In particular
the illustrations include quadratic residue calculations from a
Galois field for both the $p^1$ and $p^k$ cases, polynomial
division with remainder, and construction of the prerequisite
Jacobsthal matrix for both cases.

The construction methods are implemented in Java at
\url{https://github.com/wannamak/hadamard/}.
\end{abstract}

\section{Choosing a construction}

\begin{table}[htbp]
    \centering
    \caption{Paley Constructions through order 200}
    \begin{tblr}{colspec = { c c c @{\hskip 2cm} c c c },
                 % vlines,
                 hline{1,2,Z} = {1pt},  % <---
                 colsep=4pt,
                 }
    \thead{Hadamard\\order} & \thead{Type I\\$p^k$} & \thead{Type II\\$p^k$} &
    \thead{Hadamard\\order} & \thead{Type I\\$p^k$} & \thead{Type II\\$p^k$}\\
       4 & 3 &   &   84 & 83 & 41 \\
       8 & 7 &   &  100 &  & $7^2$ \\
      12 & 11 & 5  &  104 & 103 &  \\
      20 & 19 & $3^2$  &  108 & 107 & 53 \\
      24 & 23 &   &  124 &  & 61 \\
      28 & $3^3$ & 13  &  128 & 127 &  \\
      32 & 31 &   &  132 & 131 &  \\
      36 &  & 17  &  140 & 139 &  \\
      44 & 43 &   &  148 &  & 73 \\
      48 & 47 &   &  152 & 151 &  \\
      52 &  & $5^2$  &  164 & 163 & $3^4$ \\
      60 & 59 & 29  &  168 & 167 &  \\
      68 & 67 &   &  180 & 179 & 89 \\
      72 & 71 &   &  192 & 191 &  \\
      76 &  & 37  &  196 &  & 97 \\
      80 & 79 &   &  200 & 199 &  \\
    \end{tblr}
\end{table}

% \section*{Mission Statement}

\end{document}
