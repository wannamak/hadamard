\documentclass{article}

\usepackage{array}
\usepackage{amsmath, amssymb, amsfonts} 
\usepackage[english]{babel}
\usepackage{booktabs}
\usepackage{enumitem}
\usepackage{float}
\usepackage[margin=0.75in]{geometry}
\usepackage{graphicx}
\usepackage[colorlinks = true, linkcolor = blue, urlcolor = blue,
            citecolor = blue, anchorcolor = blue]{hyperref}
\usepackage[utf8]{inputenc}
\usepackage{makecell}
\usepackage{tabularray}
\usepackage{wrapfig}
\newcommand{\Mod}[1]{\ (\mathrm{mod}\ #1)}
\setcounter{MaxMatrixCols}{20}

\title{Paley Constructions of Hadamard Matrixes}
\author{Keith Wannamaker \\
\\
{\href{mailto:keith@wannamaker.org}{keith@wannamaker.org}}}

\begin{document}
\maketitle

\begin{abstract}
This document presents examples of constructing Hadamard matrixes
using two methods attributed to Raymond Paley.  In particular
the illustrations include quadratic residue calculations from a
Galois field for both the $p^1$ and $p^k$ cases, polynomial
division with remainder, and construction of the prerequisite
Jacobsthal matrix for both cases.

The construction methods are implemented in Java at
\url{https://github.com/wannamak/hadamard/}.
\end{abstract}

\section{Background}

A Hadamard matrix is an orthogonal matrix whose entries are -1 or 1, satisfying
\begin{equation}
\textnormal{H}\textnormal{H}^{T} = {n}\textnormal{I}_{n}
\end{equation}
For example, for n=4,
\begin{equation}
\begin{pmatrix}
1 & 1 & 1 & 1 \\
1 & 1 & -1 & -1 \\
1 & -1 & -1 & 1 \\
1 & -1 & 1 & -1 \\
\end{pmatrix}
\begin{pmatrix}
1 & 1 & 1 & 1 \\
1 & 1 & -1 & -1 \\
1 & -1 & -1 & 1 \\
1 & -1 & 1 & -1 \\
\end{pmatrix}
= 4
\begin{pmatrix}
1 & 0 & 0 & 0 \\
0 & 1 & 0 & 0 \\
0 & 0 & 1 & 0 \\
0 & 0 & 0 & 1 \\
\end{pmatrix}
\end{equation}
Properties:
\begin{itemize}
\item Any row (or column) may be exchanged with any row (or column).
\item Any row (or column) may be negated.
\item There are exactly n/2 differences between any two rows (or columns).
\end{itemize}

\section{Choosing a construction}

The Paley Construction methods for Hadamard matrices are based on a prime or a prime
power.  The resulting order is related to the prime or prime power by either:
\begin{equation}
n = p^k + 1
\end{equation}
or
\begin{equation}
n = 2 (p^k + 1)
\end{equation}
depending on the construction method.

\bigskip
Construction type one only works for the subset of odd primes where:
\begin{equation}
p\Mod{4} = 3
\end{equation}

Construction type two only works for the subset of odd primes where:
\begin{equation}
p\Mod{4} = 1
\end{equation}

Table 1 shows different alternatives for
Paley construction of order $\le 200$.  Notably absent are orders such as 16, which
cannot be constructed by Paley's methods but which can be constructed by other techniques.

\begin{table}[H]
    \centering
    \caption{Paley Constructions through order 200}
    \begin{tblr}{colspec = { c c c @{\hskip 2cm} c c c },
                 % vlines,
                 hline{1,2,Z} = {1pt},  % <---
                 colsep=4pt,
                 }
    \thead{Hadamard\\order} & \thead{Type I\\$p^k$} & \thead{Type II\\$p^k$} &
    \thead{Hadamard\\order} & \thead{Type I\\$p^k$} & \thead{Type II\\$p^k$}\\
       4 & 3 &   &   84 & 83 & 41 \\
       8 & 7 &   &  100 &  & $7^2$ \\
      12 & 11 & 5  &  104 & 103 &  \\
      20 & 19 & $3^2$  &  108 & 107 & 53 \\
      24 & 23 &   &  124 &  & 61 \\
      28 & $3^3$ & 13  &  128 & 127 &  \\
      32 & 31 &   &  132 & 131 &  \\
      36 &  & 17  &  140 & 139 &  \\
      44 & 43 &   &  148 &  & 73 \\
      48 & 47 &   &  152 & 151 &  \\
      52 &  & $5^2$  &  164 & 163 & $3^4$ \\
      60 & 59 & 29  &  168 & 167 &  \\
      68 & 67 &   &  180 & 179 & 89 \\
      72 & 71 &   &  192 & 191 &  \\
      76 &  & 37  &  196 &  & 97 \\
      80 & 79 &   &  200 & 199 &  \\
    \end{tblr}
\end{table}

\section{Quadratic residuals of primes}

Paley made the connection between the pattern of quadratic residuals and
the pattern of +1s in a Hadamard matrix.

To calculate quadratic residuals for odd $n$, for each integer 1..$(n+1)/2$,
    examine the square modulo $n$.  If non-zero, the square modulo $n$ is
    a residual.

\begin{table}[H]
    \centering
    \caption{Quadratic residues of selected odd primes}
    \begin{tblr}{colspec = { c l },
        hline{1,Z} = {1pt}}
        3 & 1\\
        5 & 1, 4\\
        7 & 1, 2, 4\\
        9 & 1, 4, 7\\
        11 & 1, 3, 4, 5, 9\\
        13 & 1, 3, 4, 9, 10, 12\\
        17 & 1, 2, 4, 8, 9, 13, 15, 16\\
        19 & 1, 4, 5, 6, 7, 9, 11, 16, 17\\
        21 & 1, 4, 7, 9, 15, 16, 18\\
        23 & 1, 2, 3, 4, 6, 8, 9, 12, 13, 16, 18\\
        25 & 1, 4, 6, 9, 11, 14, 16, 19, 21, 24\\
        29 & 1, 4, 5, 6, 7, 9, 13, 16, 20, 22, 23, 24, 25, 28\\
        31 & 1, 2, 4, 5, 7, 8, 9, 10, 14, 16, 18, 19, 20, 25, 28\\
        33 & 1, 3, 4, 9, 12, 15, 16, 22, 25, 27, 31\\
        37 & 1, 3, 4, 7, 9, 10, 11, 12, 16, 21, 25, 26, 27, 28, 30, 33, 34, 36\\
        41 & 1, 2, 4, 5, 8, 9, 10, 16, 18, 20, 21, 23, 25, 31, 32, 33, 36, 37, 39, 40\\
        43 & 1, 4, 6, 9, 10, 11, 13, 14, 15, 16, 17, 21, 23, 24, 25, 31, 35, 36, 38, 40, 41\\
        45 & 1, 4, 9, 10, 16, 19, 25, 31, 34, 36, 40\\
        47 & 1, 2, 3, 4, 6, 7, 8, 9, 12, 14, 16, 17, 18, 21, 24, 25, 27, 28, 32, 34, 36, 37, 42\\
        49 & 1, 2, 4, 8, 9, 11, 15, 16, 18, 22, 23, 25, 29, 30, 32, 36, 37, 39, 43, 44, 46\\
        53 & 1, 4, 6, 7, 9, 10, 11, 13, 15, 16, 17, 24, 25, 28, 29, 36, 37, 38, 40, 42, 43, 44, 46, 47, 49, 52\\
    \end{tblr}
\end{table}

\section{Quadratic residuals of prime powers}

For prime powers, the elements are derived from taking all possible combinations of
coefficients from the elements of the prime itself.  For example, for p=3, we consider
elements 0, 1, and 2.  So the elements of $3^2$ are:

\begin{equation}
0*x+0, 0*x+1, 0*x+2, 1*x+0, 1*x+1, 1*x+2, 2*x+0, 2*x+1, 2*x+2\\
\end{equation}

or
\begin{equation}
0,     1,     2,   x,     x+1,   x+2, 2x,   2x+1, 2x+2\\
\end{equation}

These are laid out as row and column labels, and the value of $row - column$
is calculated, modulo the minimum irreducible polynomial.

The quadratic residues for this field are:
\begin{equation}
    1, 2, x, 2x
\end{equation}

\section{Paley Construction Type One}

Construction type one assigns $row - column\Mod{p}$ to most cells, except the left column (all 1s),
the top row (all 1s), and the remaining diagonal (all -1s):

\begin{equation}
\begin{pmatrix}
    1 & 1 & 1 & 1 & 1 & 1 & 1 & 1  \\
    1 & -1 & 6 & 5 & 4 & 3 & 2 & 1 \\
    1 & 1 & -1 & 6 & 5 & 4 & 3 & 2 \\
    1 & 2 & 1 & -1 & 6 & 5 & 4 & 3 \\
    1 & 3 & 2 & 1 & -1 & 6 & 5 & 4 \\
    1 & 4 & 3 & 2 & 1 & -1 & 6 & 5 \\
    1 & 5 & 4 & 3 & 2 & 1 & -1 & 6 \\
    1 & 6 & 5 & 4 & 3 & 2 & 1 & -1 \\
\end{pmatrix}
\end{equation}

Now, replace all the $row - column\Mod{p}$ values with 1 if the number is a quadratic
residual, else -1.  For p = 7, residuals are (1, 2, 4), which yields the order 8 Hadamard matrix:

\begin{equation}
\begin{pmatrix}
    1 & 1 & 1 & 1 & 1 & 1 & 1 & 1  \\
    1 & - & - & - & 1 & - & 1 & 1 \\
    1 & 1 & - & - & - & 1 & - & 1 \\
    1 & 1 & 1 & - & - & - & 1 & - \\
    1 & - & 1 & 1 & - & - & - & 1 \\
    1 & 1 & - & 1 & 1 & - & - & - \\
    1 & - & 1 & - & 1 & 1 & - & - \\
    1 & - & - & 1 & - & 1 & 1 & - \\
\end{pmatrix}
\end{equation}

\section{Paley Construction Type Two}

This method creates four matrices similar to type one, and concatenates them into the result.
    
Consider p = 5.  We assign $row - column\Mod{p}$ values as before, with 1's in the top row,
    1's in the left column, and 0's in the diagonal, with 0 being in the top left cell:

\begin{equation}
    \begin{pmatrix}
        0 & 1 & 1 & 1 & 1 & 1 \\
        1 & 0 & 4 & 3 & 2 & 1 \\
        1 & 1 & 0 & 4 & 3 & 2 \\
        1 & 2 & 1 & 0 & 4 & 3 \\
        1 & 3 & 2 & 1 & 0 & 4 \\
        1 & 4 & 3 & 2 & 1 & 0 \\
    \end{pmatrix}
\end{equation}

Now, replace all the $row - column\Mod{p}$ values with 1 if the number is a quadratic
residual, else -1.  For p = 5, residuals are (1, 4), which yields the matrix:

\begin{equation}
    \begin{pmatrix}
        0 & 1 & 1 & 1 & 1 & 1 \\
        1 & 0 & 1 & -1 & -1 & 1 \\
        1 & 1 & 0 & 1 & -1 & -1 \\
        1 & -1 & 1 & 0 & 1 & -1 \\
        1 & -1 & -1 & 1 & 0 & 1 \\
        1 & 1 & -1 & -1 & 1 & 0 \\
    \end{pmatrix}
\end{equation}

Now, assemble 4 copies of this matrix M in the following manner:

\begin{equation}
    \begin{pmatrix}
        M + I & M - I \\
        M - I & -M - I \\
    \end{pmatrix}
\end{equation}

This constructs the following Hadamard matrix of order 12:

\begin{equation}
\begin{pmatrix}
        1 & 1 & 1 & 1 & 1 & 1 & - & 1 & 1 & 1 & 1 & 1 \\
        1 & 1 & 1 & - & - & 1 & 1 & - & 1 & - & - & 1 \\
        1 & 1 & 1 & 1 & - & - & 1 & 1 & - & 1 & - & - \\
        1 & - & 1 & 1 & 1 & - & 1 & - & 1 & - & 1 & - \\
        1 & - & - & 1 & 1 & 1 & 1 & - & - & 1 & - & 1 \\
        1 & 1 & - & - & 1 & 1 & 1 & 1 & - & - & 1 & - \\
        - & 1 & 1 & 1 & 1 & 1 & - & - & - & - & - & - \\
        1 & - & 1 & - & - & 1 & - & - & - & 1 & 1 & - \\
        1 & 1 & - & 1 & - & - & - & - & - & - & 1 & 1 \\
        1 & - & 1 & - & 1 & - & - & 1 & - & - & - & 1 \\
        1 & - & - & 1 & - & 1 & - & 1 & 1 & - & - & - \\
        1 & 1 & - & - & 1 & - & - & - & 1 & 1 & - & - \\
\end{pmatrix}
\end{equation}

\section{Code is King}

Please see the Github repository above for code which implements these construction techniques.
Requiem Aeternam for Mr. Paley who died tragically young in Banff National Park.  His mind
recognized an amazing pattern in the age before calculators and computers.

\end{document}
